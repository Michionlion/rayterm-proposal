\documentclass[11pt]{article}

\usepackage[T1]{fontenc}
\usepackage{mathptmx}
\topmargin 0.0in
\setlength{\textwidth} {420pt}
\setlength{\textheight} {620pt}
\setlength{\oddsidemargin} {20pt}
\setlength{\marginparwidth} {72in}

\usepackage{fancyhdr}
\usepackage{url}
\usepackage{graphicx}

% Use elastic spacing around the headers
\usepackage{titlesec}
\titlespacing\section{0pt}{6pt plus 4pt minus 2pt}{4pt plus 2pt minus 2pt}

% set it so that subsubsections have numbers and they
% are displayed in the TOC (maybe hard to read, might want to disable)
\setcounter{secnumdepth}{3}
\setcounter{tocdepth}{3}

% define widow protection
\def\widow#1{\vskip #1\vbadness10000\penalty-200\vskip-#1}

\clubpenalty=10000  % Don't allow orphans
\widowpenalty=10000 % Don't allow widows

% this should give you the ability to use some math symbols that
% were available by default in standard latex (i.e. \Box)
\usepackage{latexsym}

% define a little section heading that doesn't go with any number
\def\littlesection#1{
  \widow{2cm}
  \vskip 0.5cm
  \noindent{\bf #1}
  \vskip 0.0001cm
}

\pagestyle{fancyplain}

\newcommand{\tstamp}{\today}
\renewcommand{\sectionmark}[1]{\markright{#1}}
\lhead[\Section \thesection]            {\fancyplain{}{\rightmark}}
\chead[\fancyplain{}{}]                 {\fancyplain{}{}}
\rhead[\fancyplain{}{\rightmark}]       {\fancyplain{}{\thepage}}
\cfoot[\fancyplain{\thepage}{}]         {\fancyplain{\thepage}{}}

\newlength{\myVSpace}% the height of the box
\setlength{\myVSpace}{1ex}% the default,
\newcommand\xstrut{\raisebox{-.5\myVSpace}% symmetric behaviour,
  {\rule{0pt}{\myVSpace}}%
}

% leave things with no spacing extra spacing in the final version of the paper

\renewcommand{\baselinestretch}{1.0} % must go before the begin of doc

% suppress the use of indentation for a paragraph
\setlength{\parindent}{0.0in}
\setlength{\parskip}{0.1in}

\begin{document}

% handle widows appropriately

\def\widow#1{\vskip #1\vbadness10000\penalty-200\vskip-#1}

% build the title section

\makeatletter

\def\maketitle{
  \thispagestyle{empty}
  \begin{center}
    {\Huge \@title\par}
    {\normalsize \@author\par}
    \vskip .4in
  \end{center}
}

\makeatother

%   ********************************************************************
%   * Here is the first place where you need to begin customizing:     *
%   * Enter you name, the title of your proposal, etc., in the places  *
%   * indicated by the comment "% CHANGE!".                            *
%   ********************************************************************

\vspace*{-1.1in}
\title{Title of Your Senior Thesis Proposal}  % CHANGE!

% build the author section, making appropriate CHANGES
\author{
  Your Full Name\\  % CHANGE!
  Department of Computer Science\\
  Allegheny College \\
  {\tt youremail@allegheny.edu}  \\  % CHANGE!
  \url{http://www.cs.allegheny.edu/~yourwebsite/} \\   % CHANGE!
  \vspace*{.1in} \today \\ \vspace*{.1in}
}

\maketitle

% Default "abstract" environment is too small; customize one instead:
\begin{center}
  \large\bf Abstract
  \vspace{-1em}  % Reduce space between header and the abstract
\end{center}

%   ********************************************************************
%   * Here is the second place where you need to customize:            *
%   * enter your abstract in the "quote" environment:                  *
%   ********************************************************************

\begin{quote}

  Provide a concise summary of your proposed research. Remember that the abstract
  is {\it not\/} an introduction, it is a {\it summary\/} of the entire document.
  It makes sense to wait to write the abstract until the rest of the document has
  been written.

\end{quote}

\section{Introduction}
\label{sec:introduction}

% Provide an intuitive motivation for and introduction to your proposed senior
% thesis research. Whenever possible, you should use one or more concrete examples
% and technical diagrams.

\section{Related Work}
\label{sec:relatedwork}

% Summarize the previously published papers and books that are related to your
% proposed research. Whenever possible, you should compare and contrast your
% approach with the ones that have been discussed in the past. As you describe
% related papers, please make sure that you cite them
% properly~\cite{conrad-gecco-selection-study}.

\section{Method of Approach}
\label{sec:method}

% Use technical diagrams, equations, algorithms, and paragraphs of text to
% describe the research that you intend to complete. See the \LaTeX\ source file
% for the proposal to learn how Figure~\ref{intro-fig1} and Table~\ref{intro-tab1}
% were included. Be sure to number all figures and tables and to explicitly refer
% to them in your text.

\section{Evaluation Strategy}
\label{sec:evaluate}

% Explain what steps you will take to evaluate your proposed method. If you intend
% to conduct experiments, then you must clearly define your evaluation metrics.

\section{Research Schedule}
\label{sec:schedule}

% Identify the main phases and tasks of your research project and set deadlines
% for when you will be able to complete each of these items.

\section{Conclusion}
\label{sec:conclusion}

% Provide a summary of your proposed research and suggest the impact that it may
% have on the discipline of computer science. If possible, you may also suggest
% some areas for future research.

\bibliographystyle{plain}
\bibliography{senior_thesis_proposal}
\end{document}
